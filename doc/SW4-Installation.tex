%-*-LaTeX-*-
\documentclass[11pt]{article}
%%%%%%%%%%%%%%%%%%%%%%%%%%%%%%%%%%%%%%%%
% Use the approved methods for setting lengths
%\oddsidemargin  0.0in
%\evensidemargin 0.0in
%\textwidth      6.5in
%\textheight     9.0in
\setlength{\oddsidemargin}{0.0in}
\setlength{\evensidemargin}{0.0in}
\setlength{\textwidth}{6.5in}
\setlength{\textheight}{8.7in}
% Based on document style, and taller text body height, set
% weird LaTex margin/header (box heights).  This fixes
% the disappearing page numbers (which never disappeared; they
% just got printed beyond the borders of the physical paper)
\setlength{\voffset}{0pt}
\setlength{\topmargin}{0pt}
\setlength{\headheight}{10pt}
\setlength{\headsep}{25pt}
%%%%%%%%%%%%%%%%%%%%%%%%%%%%%%%%%%%%%%%%
\usepackage{color}
\usepackage{textpos}
%\usepackage{html}
\usepackage{makeidx}
%\usepackage{times}
\usepackage{hyperref}
\usepackage{verbatim}
\usepackage{graphicx}
%\usepackage{eufrak}
\usepackage{amsmath}

\makeindex

\tolerance=600

\newcommand{\Fb}{{\bf F}}
\newcommand{\Lb}{{\bf L}}
\newcommand{\gb}{{\bf g}}
\newcommand{\nb}{{\bf n}}
\newcommand{\ub}{{\bf u}}
\newcommand{\xb}{{\bf x}}
\newcommand{\p}{\partial}
\renewcommand{\div}{{\rm div}}
\renewcommand{\arraystretch}{1.3}

\begin{document}

\title{Installing SW4 version 1.1}

\author{ N. Anders Petersson$^1$ \and Bj\"orn Sj\"ogreen\thanks{Center for Applied Scientific
    Computing, Lawrence Livermore National Laboratory, PO Box 808, Livermore CA 94551. This work
    performed under the auspices of the U.S. Department of Energy by Lawrence Livermore National
    Laboratory under contract DE-AC52-07NA27344. This is contribution LLNL-SM-662989.} \and Eric
  Heien\thanks{Computational Infrastructure for Geodynamics, University of California, Davis.}}
\date{October 16, 2014} \maketitle

%\thispagestyle{plain}

%\pagebreak


%\pagebreak
\tableofcontents

\section{Introduction}
The sole purpose of this document is to describe the installation process of the seismic wave
propagation code \emph{SW4}. A comprehensive user's guide of this code is provided in the report by
Petersson and Sjogreen~\cite{SW4-11}.

\section{Compilers and third party libraries}

Before you can build \emph{SW4} on your system, you must have
\begin{enumerate}
\item the \verb+lapack+ and \verb+blas+ libraries. These libraries provide basic linear algebra
  functionality and are pre-installed on many machines;
\item a MPI-2 library. This library provides support for message passing on parallel
  machines. Examples of open source implementations include Mpich-2 and OpenMPI. Note that the MPI-2
  library must be installed even if you are only building \emph{SW4} for a single core system.
\end{enumerate}
To avoid incompatibility issues and linking problems, we recommend using the same compiler for
the libraries as for \emph{SW4}.

For a complete installation that supports projections from the Proj.4 library and material models
from an e-tree database, you need to download and install three additional libraries:
\begin{itemize}
\item The Proj.4 library, {\tt http://trac.osgeo.org/proj}
\item The Euclid e-tree library, {\tt http://www-2.cs.cmu.edu/~euclid}
\item The cencalvm library, \\{\tt http://earthquake.usgs.gov/data/3dgeologic/cencalvm\_doc/index.html}
\end{itemize}
These libraries need to be installed under the same directory, such that each library installs its
files in the {\tt lib} and {\tt include} sub-directories. See Section~\ref{sec:cencalvm-install} for
details.

\paragraph{Mac computers}

We recommend using the MacPorts package manager for installing the required compilers and
libraries. Simply go to www.macports.org, and install macports on your system. With that in place,
you can use the \verb+port+ command as follows
\begin{verbatim}
shell> sudo port install gcc48
shell> sudo port select --set gcc mp-gcc48
shell> sudo port install openmpi +gcc48
shell> sudo port select --set mpi openmpi-gcc48-fortran
\end{verbatim}
Here, \verb+gcc48+ refers to version 4.8 of the Gnu compiler suite. Compiler versions are bound to
change in the future, so the above commands will need to be modified accordingly. Before starting,
make sure you install a version of gcc that is compatible with the MPI library package. The above
example installs the \verb+openmpi+ package with the \verb+gcc48+ variant, which includes a
compatible Fortran compiler. Alternatively, you can use the \verb+mpich2+ package. Note that the
\verb+port select+ commands are used to create shortcuts to the compilers and MPI environment. By
using the above setup, the Gnu compilers can be accessed with \verb+gcc+ and \verb+gfortran+
commands, and the MPI compilers and execution environment are called \verb+mpicxx+, \verb+mpif77+,
and \verb+mpirun+, respectively.

The \verb+lapack+ and \verb+blas+ libraries are preinstalled on recent Macs and can be accessed
using the \verb+-framework vecLib+ link option. If that is not available or does not work on your
machine, you can download \verb+lapack+ and \verb+blas+ from www.netlib.org.

\paragraph{Linux machines}
We here give detailed instructions for installing the third part libraries under 64 bit, Fedora Core
18 Linux. Other Linux variants use similar commands for installing software packages, but note that 
the package manager \verb+yum+ is specific to Fedora Core.

You need to have root privileges to install precompiled packages. Start by opening an xterm and
set your user identity to root by the command
\begin{verbatim}
su -
\end{verbatim}
Install the compilers by issuing the commands
\begin{verbatim}
yum install gcc
yum install gcc-c++
yum install gcc-gfortran
\end{verbatim}
You install the \verb+mpich2+ library and include files with the command
\begin{verbatim}
yum install mpich2-devel
\end{verbatim}
The executables and libraries are installed in \verb+/usr/lib64/mpich2/bin+ and 
\verb+/usr/lib64/mpich2/lib+ respectively. We suggest that you add \verb+/usr/lib64/mpich2/bin+ to 
your path. This is done with the command
\begin{verbatim}
export PATH=${PATH}:/usr/lib64/mpich2/bin
\end{verbatim}
if your shell is \verb+bash+. For \verb+tcsh+ users, the command is
\begin{verbatim}
setenv PATH ${PATH}:/usr/lib64/mpich2/bin
\end{verbatim}
It is convenient to put the path setting command in
your startup file, \verb+.bashrc+ or \verb+.cshrc+., for bash or csh/tcsh respectively.

The \verb+blas+ and \verb+lapack+ libraries are installed with
\begin{verbatim}
yum install blas
yum install lapack
\end{verbatim}
On our system, the libraries were installed in \verb+/usr/lib64+ as \verb+libblas.so.3+ and
\verb+liblapack.so.3+. For some unknown reason, the install program does not add links to these
files with extension \verb+.so+, which is necessary for the linker to find them. We must therefore
add the links explicitly. If the libraries were installed elsewhere on your system, but you don't
know where, you can find them with the following command:
\begin{verbatim}
find / -name "*blas*" -print
\end{verbatim}
After locating the directory where the libraries reside (in this case \verb+/usr/lib64+), we add
links to the libraries with the commands:
\begin{verbatim}
cd /usr/lib64
ln -s libblas.so.3 libblas.so
ln -s liblapack.so.3 liblapack.so
\end{verbatim}
Note that you need to have \verb+root+ privileges for this to work.

%%%%%%%%%%%%%%%%%%%%%%%%%%%%%%%%%%%%%%%%%%%%%%%%
\section{Unpacking the source code tar ball}
\index{installation!directories}
%%%%%%%%%%%%%%%%%%%%%%%%%%%%%%%%%%%%%%%%%%%%%%%%

To unpack the \emph{SW4} source code, you place the file \verb+sw4-v1.1.tgz+ in the
desired directory and issue the following command:
\begin{verbatim}
shell> tar xzf sw4-v1.1.tgz
\end{verbatim}
As a result a new sub-directory named \verb+sw4-v1.1+ is created. It contains several files
and sub-directories:
%
\begin{itemize}
\item \verb+LICENSE.txt+ License information.
\item \verb+INSTALL.txt+ Information about how to build \emph{SW4} (short version).
\item \verb+README.txt+ General information about \emph{SW4}.
\item \verb+configs+ Directory containing \verb+make+ configuration files.
\item \verb+src+ C++ and Fortran source code of \emph{SW4}.
\item \verb+tools+ Matlab/Octave scripts for post processing and analysis.
\item \verb+examples+ Sample input files.
\item \verb+Makefile+ Main makefile (don't change this file!).
\item \verb+wave.txt+ Text for printing the "SW4 Lives" banner at the end of a successful build.
\end{itemize}

%%%%%%%%%%%%%%%%%%%%%%%%%%%%%%%%%%%%%%%%%%%%%%%%%%%%%%%
\section{Installing \emph{SW4} with make}\label{cha:installing-sw4}
%%%%%%%%%%%%%%%%%%%%%%%%%%%%%%%%%%%%%%%%%%%%%%%%%%%%%%%

%The \emph{SW4} source code is released under the GNU general public license and can be downloaded
%from:
%\begin{verbatim}
%https://computation.llnl.gov/casc/serpentine/software.html
%\end{verbatim}

The classical way of building \emph{SW4} uses \verb+make+. We recommend using GNU make, sometimes called
\verb+gmake+. You can check the version of make on you system with the command
\begin{verbatim}
shell> make -v
\end{verbatim}
If you don't have GNU make installed on your system, you can obtain it from www.gnu.org.

We have built \emph{SW4} and its supporting libraries on Intel based laptops and desktops running
LINUX and OSX. It has also been built on several supercomputers such as the Intel machines {\tt cab}
(at LLNL) and {\tt edison} (at LBNL), as well as the IBM BGQ machine {\tt vulcan} at LLNL. We have
successfully used the following versions of Gnu, Intel, and IBM compilers:
\begin{verbatim}
Gnu:   g++/gcc/gfortran      versions 4.5 to 4.8
Intel: icpc/icc/ifort        version 12.1
IBM Blue Gene: xlc++/xlc/xlf version 12.1
\end{verbatim}

\emph{SW4} uses the message passing interface (MPI) standard (MPI-2 to be specific) for
communication on parallel distributed memory machines. Note that the MPI library often includes
wrappers for compiling, linking, and running of MPI programs. For example, the {\tt mpich2} package
includes the {\tt mpicxx} and {\tt mpif77} compilers, as well as the {\tt mpirun} script. We highly
recommend using these programs for compiling, linking, and running \emph{SW4}.


\subsection{Basic compilation and linking of \emph{SW4}}\label{sec:basic-install}
\index{installation!basic} 

The best way of getting started is to first build \emph{SW4} without the proj.4 and cencalvm
libraries. This process should be very straight forward and the resulting \emph{SW4} executable
will support all commands except \verb+efile+ and the \verb+proj+/\verb+ellps+ options in the
\verb+grid+ command. If you need to use these options, you can always recompile \emph{SW4} after the
proj.4 and cencalvm libraries have been installed. See \S~\ref{sec:cencalvm-install} for details.

The basic build process is controlled by the environmental variables \verb+FC+, \verb+CXX+,
\verb+EXTRA_FORT_FLAGS+, \verb+EXTRA_CXX_FLAGS+, and \verb+EXTRA_LINK_FLAGS+. These variables should
hold the names of the Fortan-77 and C++ compilers, and any extra options that should be passed to
the compilers and linker. The easiest way of assigning these variables is by creating a file in the
\verb+configs+ directory called \verb+make.inc+. The \verb+Makefile+ will look for this file and
read it if it is available. There are several examples in the \verb+configs+ directory, e.g.
\verb+make.osx+ for Macs and \verb+make.linux+ for Linux machines. You should copy one of these files
to your own \verb+make.inc+ and edit it as needed. 

\subsubsection{Mac machines}
If you are on a Mac, you could copy the setup from \verb+make.osx+,
\begin{verbatim}
shell> cd configs
shell> cp make.osx make.inc
shell> cat make.inc
FC = openmpif77
CXX = openmpicxx
EXTRA_FORT_FLAGS = -fno-underscoring
EXTRA_LINK_FLAGS = -framework vecLib -L/opt/local/lib/gcc47 -lgfortran
\end{verbatim}
In this case, the \verb+blas+ and \verb+lapack+ libraries are assumed to be provided by the
\verb+-framework vecLib+ option. The \verb+libgfortran+ library is located in the directory
\verb+/opt/local/lib/gcc47+, which is where \verb+macports+ currently installs it.

\subsubsection{Linux machines}
If you are on a Linux machine, we suggest you copy the configuration options from \verb+make.linux+,
\begin{verbatim}
shell> cd configs
shell> cp make.linux make.inc
shell> cat make.inc
FC = gfortran
CXX = mpicxx
EXTRA_LINK_FLAGS = -L/usr/lib64 -llapack -lblas -lgfortran
\end{verbatim}
This setup assumes that the \verb+blas+ and \verb+lapack+ libraries are located in the
\verb+/usr/lib64+ directory. 
In the case of Fedora Core 18, we needed to set the link flag variable to
\begin{verbatim}
EXTRA_LINK_FLAGS = -Wl,-rpath=/usr/lib64/mpich2/lib  -llapack -lblas -lgfortran
\end{verbatim}

\subsubsection{Using make}
You build \emph{SW4} with the "make" command from the main directory.
\begin{verbatim}
shell> cd /enter/your/path/sw4-v1.1
shell> make
\end{verbatim}
If all goes well, you will see the SW4 Lives banner on your screen after the compilation and linking
has completed,
{\samepage
\begin{verbatim}
    
``'-.,_,.-'``'-.,_,.='``'-.,_,.-'``'-.,_,.='````'-.,_,.-'``'-.,_,.='``


  _________    ____      __      ____    _    __
 /   ____  \   \   \    /  \    /   /   / |  |  |  
 |  |    \./    \   \  /    \  /   /   /  |  |  |
 |  |______      \   \/      \/   /   /   '--'  |
 \______   \      \              /    |______   |
        |  |       \     /\     /            |  |      
 /`\____|  |        \   /  \   /             |  |      
 \_________/         \_/    \_/              |__|      
                                       
   __       __  ____    ____  _______    ______    __  
  |  |     |  | \   \  /   / |   ____|  /    __|  |  | 
  |  |     |  |  \   \/   /  |  |__     |   (__   |  | 
  |  |     |  |   \      /   |   __|    \__    |  |  | 
  |  `----.|  |    \    /    |  |____    __)   |  |__| 
  |_______||__|     \__/     |_______|  (_____/   (__)
     

``'-.,_,.-'``'-.,_,.='``'-.,_,.-'``'-.,_,.='````'-.,_,.-'``'-.,_,.='``

\end{verbatim}
}
By default, \verb+make+ builds an optimized \verb+sw4+ executable. It is located in 
\begin{verbatim}
/enter/your/path/sw4-v1.1/optimize/sw4
\end{verbatim}
You can also build an executable with debugging symbols by adding the \verb+debug=yes+ option to \verb+make+,
\begin{verbatim}
shell> cd /enter/your/path/sw4-v1.1
shell> make debug=yes
\end{verbatim}
In this case, the executable will be located in
\begin{verbatim}
/enter/your/path/sw4-v1.1/debug/sw4
\end{verbatim}
It can be convenient to add the corresponding directory to your \verb+PATH+ environment
variable. This can be accomplished by modifying your shell configuration file, e.g. \verb+~/.cshrc+
if you are using C-shell.

\subsubsection{How do I setup the {\tt make.inc} file?}

The input file for \verb+make+ is
\begin{verbatim}
sw4-v1.1/Makefile
\end{verbatim}
Do {\em not} change this \verb+Makefile+. It should only be necessary to edit your configuration
file, that is,
\begin{verbatim}
/my/path/sw4-v1.1/configs/make.inc
\end{verbatim}
Note that you must create this file, for example by copying one of the \verb+make.xyz+ files in the
same directory. The \verb+make.inc+ file holds all information that is particular for your system,
such as the name of the compilers, the location of the third party libraries, and any extra
arguments that should be passed to the compiler or linker. This file also tells \verb+make+ whether
or not the \verb+cencalvm+ and \verb+proj.4+ libraries are available and where they are located.

The following \verb+make.inc+ file includes all configurable options:
\begin{verbatim}
etree = no
SW4ROOT = /Users/petersson1

CXX = mpicxx
FC  = mpif77

EXTRA_CXX_FLAGS = -DUSING_MPI
EXTRA_FORT_FLAGS = -fno-underscoring
EXTRA_LINK_FLAGS = -framework vecLib
\end{verbatim}
The \verb+etree+ variable should be set to \verb+yes+ or \verb+no+, to indicate whether or not the
cencalvm and related libraries are available. The \verb+SW4ROOT+ variable is only used when
\verb+etree=yes+. The \verb+CXX+ and \verb+FC+ variables should be set to the names of the C++ and
Fortran compilers, respectively. Finally, the \verb+EXTRA_CXX_FLAGS+, \verb+EXTRA_FORT_FLAGS+, and
\verb+EXTRA_LINK_FLAGS+ variables should contain any additional arguments that need to be passed to
the C++ compiler, Fortran compiler, or linker, on your system.


\section{Installing the cencalvm, proj.4, and euclid libraries}\label{sec:cencalvm-install}
\index{installation!cencalvm}
\index{installation!efile}

The cencalvm library was developed by Brad Aagaard at USGS. Instructions for building the cencalvm
library as well as downloading the Etree database files for Northern California, can
currently be downloaded from
\begin{verbatim}
http://earthquake.usgs.gov/data/3dgeologic/cencalvm_doc/INSTALL.html
\end{verbatim}
The installation process for cencalvm, which is outlined below, is described in detail on the above
web page.  Note that three libraries need to be installed: euclid (etree), proj4, and cencalvm.  In
order for \emph{SW4} to use them, they should all be installed in the same base directory and you
should assign that directory to the \verb+SW4ROOT+ variable. When these three packages are installed, the
corresponding include and library files should be in the directories \verb+${SW4ROOT}/include+ and
\verb+${SW4ROOT}/lib+, respectively.

Note that the euclid library must be installed manually by explicitly copying all include files to
the include directory and all libraries to the lib directory,
\begin{verbatim}
shell> cd euclid3-1.2/libsrc
shell> make
shell> cp *.h ${SW4ROOT}/include
shell> cp libetree.* ${SW4ROOT}/lib
\end{verbatim}
The proj4 library should be configured to be installed in \verb+${SW4ROOT}+. This is accomplished by
\begin{verbatim}
shell> cd proj-4.7.0
shell> configure --prefix=${SW4ROOT}
shell> make
shell> make install
\end{verbatim}
The cencalvm library should also be configured to be installed in \verb+${SW4ROOT}+. You also have to help
the configure script finding the include and library files for the proj4 and etree libraries,
\begin{verbatim}
shell> cd cencalvm-0.6.5
shell> configure --prefix=${SW4ROOT} CPPFLAGS="-I${SW4ROOT}/include" \
                 LDFLAGS="-L${SW4ROOT}/lib"
shell> make
shell> make install
\end{verbatim}

To verify that the libraries have been installed properly, you should go to the \verb+SW4ROOT+ directory
and list the lib subdirectory (\verb+cd ${SW4ROOT}; ls lib+). You should see the following files (on
Mac OSX machines, the .so extension is replaced by .dylib ):
\begin{verbatim}
shell> cd ${SW4ROOT}
shell> ls lib
libetree.so libetree.a 
libproj.so libproj.a libproj.la 
libcencalvm.a libcencalvm.la libcencalvm.so 
\end{verbatim}
Furthermore, if you list the include subdirectory, you should see include files such as 
\begin{verbatim}
shell> cd ${SW4ROOT} %$
shell> ls include
btree.h etree.h etree_inttypes.h
nad_list.h projects.h proj_api.h
cencalvm
\end{verbatim}
Note that the include files for cencalvm are in a subdirectory with the same name.

\subsection{Building {\tt sw4} with efile and proj.4 support}
Once you have successfully installed the euclid, proj.4 and cencalvm libraries, it should be easy to
re-configure \emph{SW4} to use them. Simply edit your configuration file (\verb+make.inc+)
by adding two lines to the top of the file:
\begin{verbatim}
etree = yes
SW4ROOT = /thid/party/basedir
...
\end{verbatim}
You then need to re-compile \emph{SW4}. Go to the \emph{SW4} main directory, clean up the previous object
files and executable, and re-run make:
\begin{verbatim}
shell> cd /my/installation/dir/sw4-v1.1
shell> make clean
shell> make
\end{verbatim}
If all goes well, the ``SW4 lives'' banner is shown after the make command is
completed. As before, the \verb+sw4+ executable will be located in the \verb+optimize+ or
\verb+debug+ directories.

\section{Disclaimer} 
This document was prepared as an account of work sponsored by an agency of the United States
government. Neither the United States government nor Lawrence Livermore National Security, LLC, nor
any of their employees makes any warranty, expressed or implied, or assumes any legal liability or
responsibility for the accuracy, completeness, or usefulness of any information, apparatus, product,
or process disclosed, or represents that its use would not infringe privately owned
rights. Reference herein to any specific commercial product, process, or service by trade name,
trademark, manufacturer, or otherwise does not necessarily constitute or imply its endorsement,
recommendation, or favoring by the United States government or Lawrence Livermore National Security,
LLC. The views and opinions of authors expressed herein do not necessarily state or reflect those of
the United States government or Lawrence Livermore National Security, LLC, and shall not be used for
advertising or product endorsement purposes. 

\bibliographystyle{plain}

\bibliography{refs} 

\end{document}
