%!TEX root = SISC_elastic_3d.tex
\subsection{Gaussian source}\label{gaussian_source}
In this section, we show that there is no obvious artifacts are generated by the curvilinear interface. For the coarse domain $\Omega^c$, the mapping is given by
\[ {\bf x} = {\bf X}^c({\bf r}) = \left(\begin{array}{c}
2000 r^{(1)}\\
2000 r^{(2)}\\
r^{(3)} \theta_i\big(r^{(1)}, r^{(2)}\big) + (1-r^{(3)}) \theta_b\big(r^{(1)},r^{(2)}\big) \end{array}\right). \]
Here, $0\leq r^{(1)}, r^{(2)}, r^{(3)}\leq 1$, $\theta_i$ represents the interface surface geometry,
\begin{equation}\label{interface_gausian}
\theta_i\big(r^{(1)},r^{(2)}\big) = 800+20\sin(4\pi r^{(1)})+20\cos(4\pi r^{(2)}),
\end{equation}
and $\theta_b$ is the bottom surface geometry,
\begin{equation*}
\theta_b\big(r^{(1)},r^{(2)}\big) = 0.
\end{equation*}
As for the fine domain $\Omega^f$, the mapping is chosen to be
\[ {\bf x} = {\bf X}^f({\bf r}) = \left(\begin{array}{c}
2000 r^{(1)}\\
2000 r^{(2)}\\
r^{(3)}\theta_t\big(r^{(1)},r^{(2)}\big) + (1-r^{(3)})\theta_i\big(r^{(1)},r^{(2)}\big)\end{array}\right), \]
where $0\leq r^{(1)}, r^{(2)}, r^{(3)}\leq 1$ and $\theta_t$ is the top surface geometry with
\begin{equation*}
\theta_t\big(r^{(1)},r^{(2)}\big) = 1000,
\end{equation*}
$\theta_i$ is the interface geometry which is given in (\ref{interface_gausian}). For both fine domain and coarse domain, let the density vary according to
\begin{equation*}
\rho(x^{(1)},x^{(2)},x^{(3)}) = 1.5\times 10^3,
\end{equation*}
and material parameters $\mu, \lambda$ satisfy
\begin{equation*}
\mu(x^{(1)},x^{(2)},x^{(3)}) = 1.5\times 10^9,\ \ 
\lambda(x^{(1)},x^{(2)},x^{(3)})  = 3\times 10^9,
\end{equation*}
respectively. We also impose a Gaussian source at the top surface
\[{\bf g} = (g_1,g_2,g_3)^T ,\]
where, $g_1 = g_2 = 0$, and 
\[g_3 = 10^9 \text{exp}\left(-\left(\frac{t-4/44.2}{1/44.2}\right)^2\right)\text{exp}\left(-\left(\frac{x^{(1)}-1000}{12.5}\right)^2-\left(\frac{x^{(2)}-1000}{12.5}\right)^2\right).\]

Homogeneous Dirichlet boundary conditions are imposed at other boundaries. The external forcing is chosen to be zero everywhere and the initial conditions are also set to be zero everywhere, ${\bf F}(\cdot,0) = {\bf C}(\cdot,0) = {\bf u}(\cdot,0) = {\bf 0}, {\bf u}(\cdot,t) = (u_1(\cdot,t), u_2(\cdot,t), u_3(\cdot,t))^T$.

For the reference solution, we compute it on a Cartesian mesh without a mesh refinement interface. More precisely, we have $n_1 = n_2 = 201, n_3 = 101$. In the experiments for the curvilinear interface with mesh refinement, we have $n_1^{2h} = n_2^{2h} = 101, n_3^{2h} = 41$ and $n_1^h = n_2^h = 201, n_3^h = 21$. The final simulation time is $T = 0.4$.

\begin{figure}[htbp]
	\centering
	\includegraphics[width=0.49\textwidth,trim={0 2.8cm 0 2.8cm}, clip]{u1_t02_cartesian.png}
	\includegraphics[width=0.49\textwidth,trim={0 2.8cm 0 2.8cm}, clip]{u1_t02_curvi_mr.png}\\
	\includegraphics[width=0.49\textwidth,trim={0 2.8cm 0 2.8cm}, clip]{u1_t04_cartesian.png}
	\includegraphics[width=0.49\textwidth,trim={0 2.8cm 0 2.8cm}, clip]{u1_t04_curvi_mr.png}
\caption{The graphs for $u_1$. In the two figures, on the left, we show reference solutions at $t=0.2$ and $t=0.4$, computed on a Cartesian mesh without a mesh refinement interface. On the right, the two figures show the solutions computed on a curvilinear mesh with a curved mesh refinement interface at $t=0.2$ and $t=0.4$. The curved interfaces are marked with the red dash lines. Note that $x,z$ in the graph correspond to $x^{(1)}, x^{(3)}$ respectively.}
\label{u1}
\end{figure}

%\begin{figure}[htbp]
%	\centering
%	\includegraphics[width=0.4\textwidth,trim={0 2.8cm 0 2.8cm}, clip]{u2_t02_cartesian.png}
%	\includegraphics[width=0.4\textwidth,trim={0 2.8cm 0 2.8cm}, clip]{u2_t02_curvi_mr.png}\\
%	\includegraphics[width=0.4\textwidth,trim={0 2.8cm 0 2.8cm}, clip]{u2_t04_cartesian.png}
%	\includegraphics[width=0.4\textwidth,trim={0 2.8cm 0 2.8cm}, clip]{u2_t04_curvi_mr.png}
%	\caption{The graph for $u_2$. From left to right are for Cartesian mesh without mesh refinement interface and curvilinear mesh with mesh refinement interface respectively. From top to bottom are for $t = 0.2$ and $t = 0.4$ respectively. Note that $x,z$ in the graph correspond to $x^{(1)}, x^{(3)}$ respectively.}\label{u2}
%\end{figure}

\begin{figure}[htbp]
	\centering
	\includegraphics[width=0.49\textwidth,trim={0 2.8cm 0 2.8cm}, clip]{u3_t02_cartesian.png}
	\includegraphics[width=0.49\textwidth,trim={0 2.8cm 0 2.8cm}, clip]{u3_t02_curvi_mr.png}\\
	\includegraphics[width=0.49\textwidth,trim={0 2.8cm 0 2.8cm}, clip]{u3_t04_cartesian.png}
	\includegraphics[width=0.49\textwidth,trim={0 2.8cm 0 2.8cm}, clip]{u3_t04_curvi_mr.png}
	\caption{The graphs for $u_3$. In the two figures, on the left, we show reference solutions at $t=0.2$ and $t=0.4$, computed on a Cartesian mesh without a mesh refinement interface. On the right, the two figures show the solutions computed on a curvilinear mesh with a curved mesh refinement interface at $t=0.2$ and $t=0.4$. The curved interfaces are marked with the red dash lines. Note that $x,z$ in the graph correspond to $x^{(1)}, x^{(3)}$ respectively.}
\label{u3}
\end{figure}
From Figure \ref{u1} and Figure \ref{u3}, we observe that there is no significant reflection at the mesh refinement interface for both component $u_1$ and $u_3$. As for component $u_2$, it is zero up to round-off error and is not presented here.