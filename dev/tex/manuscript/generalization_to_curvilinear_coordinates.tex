%!TEX root = elastic_3d_sbp.tex
\subsection{Generalization to curvilinear coordinates}
%Present the transformed equation. Notation is important. We can write a few formulas first, and discuss if it is good notation. Maybe we shall follow notations from one of Anders' papers?

In this section, we consider a curvilinear domain $\Omega$. Assume there is a one-to-one mapping ${\bf x} = {\bf x}({\bf r}) : [0,1]^3 \rightarrow \Omega \subset \mathbb{R}^3$ with ${\bf x}({\bf r}) = (x^{(1)}({\bf r}),x^{(2)}({\bf r}),x^{(3)}({\bf r}))^T, {\bf r} = (r^{(1)},r^{(2)},r^{(3)})^T, 0\leq r^{(i)}\leq1, i = 1,2,3$. We define the derivative of the forward mapping to be 
\begin{equation}\label{elastic_eq_carte}
{\bf a}_k := \bar{\partial}_k{\bf x} = \left(\frac{\partial x^{(1)}}{\partial r^{(k)}},\frac{\partial x^{(2)}}{\partial r^{(k)}},\frac{\partial x^{(3)}}{\partial r^{(k)}}\right)^T,
\end{equation}
for $k = 1,2,3$ and the backward mapping to be
\begin{equation*}
{\bf a}^k := \nabla r^{(k)} = \left(\frac{\partial r^{(k)}}{\partial x^{(1)}},\frac{\partial r^{(k)}}{\partial x^{(2)}},\frac{\partial r^{(k)}}{\partial x^{(3)}}\right)^T := (\xi_{1k},\xi_{2k},\xi_{3k})^T,
\end{equation*}
for $k = 1,2,3$. It is known that the backward mapping can be expressed by the forward mapping \cite{?}, 
\begin{equation*}
{\bf a}^i = \frac{1}{J}({\bf a}_j\times{\bf a}_k), \ \ \ (i,j,k) \ \text{cycle}.
\end{equation*}
Here, $J:=\text{det}({\bf a}_1, {\bf a}_2, {\bf a}_3)$ is the Jacobian of the forward mapping. We assume that the mapping is non-singular such that  $0<J<\infty$.

After applying the forward and backward mapping, reader can refer to \cite{?} for details, the elastic wave equation (\ref{elastic_eq_carte}) can be written as
\begin{equation}\label{elastic_eq_curvi}
\rho\frac{\partial^2 {\bf u}}{\partial t^2} = \frac{1}{J}\left[\bar{\partial }_1(\bar{A}_1\bar{\nabla}{\bf u}) + \bar{\partial }_2(\bar{A}_2\bar{\nabla}{\bf u}) +\bar{\partial }_3(\bar{A}_3\bar{\nabla}{\bf u}) \right] + {\bf F},
\end{equation}
where
\begin{align*}
\bar{A}_1\bar{\nabla}{\bf u} &:= N_{11}\bar{\partial}_1{\bf u} + N_{12}\bar{\partial}_2{\bf u} + N_{13}\bar{\partial}_3{\bf u}, \\
\bar{A}_2\bar{\nabla}{\bf u} &:= N_{21}\bar{\partial}_1{\bf u} + N_{22}\bar{\partial}_2{\bf u} + N_{23}\bar{\partial}_3{\bf u}, \\
\bar{A}_3\bar{\nabla}{\bf u} &:= N_{31}\bar{\partial}_1{\bf u} + N_{32}\bar{\partial}_2{\bf u} + N_{33}\bar{\partial}_3{\bf u},
\end{align*}
here, 
\begin{equation}\label{definition_Nij}
N_{ij} = J\bar{P}_i^TC\bar{P}_j \ \ \text{with} \ \ \bar{P}_i = \sum_{j=1}^3\xi_{ji}P_j.
\end{equation}
The definitions of matrices $C$ and $P_i, i = 1,2,3$ can be found in Appendix ?. The matrices $N_{ij}$ have similar properties as the corresponding matrices $M_{ij}$ in the Cartesian case \eqref{Mmatrices}, that is $N_{ii}$ are symmetric positive definite, and $N_{ij}=N_{ji}^T$ for $i,j=1,2,3$ with $i\ne j$.

The transformation of Dirichlet boundary condition between parameter coordinates and curvilinear coordinates is straightfroward, ${\bf u}({\bf x}) = {\bf u}(\bf r)$. To transfer the boundary forcing condition from curvilinear coordinates to parameter coordinates, we firstly write the unit outside normal to be the function of metric derivatives, for example, along the boundary $r^{(3)} = 1$ or $r^{(3)} = 0$,
\begin{equation}\label{definition_bdry3_curvi}
{\bf n}_3^{\pm} := (n^{\pm,(1)}_3,n^{\pm,(2)}_3,n^{\pm,(3)}_3)^T = \pm \frac{\nabla r^{(3)}}{\left|\nabla r^{(3)}\right|} = \frac{\pm 1}{\sqrt{((\xi_{13})^2+(\xi_{23})^2+(\xi_{33})^2)}}\left(\xi_{13},\xi_{23},\xi_{33}\right)^T,
\end{equation}
here, $'+'$ corresponds to $r^{(3)} = 1$ and $'-'$ corresponds to $r^{(3)} = 0$. Then after some straightforward calculations, the boundary forcing condition can be written as
\begin{equation}\label{bdry3_curvi}
\mathcal{T}\cdot{\bf n}_3^{\pm} = \frac{\pm 1}{J\left|\nabla r^{(3)}\right|}\bar{A}_3\bar{\nabla}{\bf u}, \ \ \ r^{(3)} = 1, 0.
\end{equation}
Similarly, we can derive the boundary forcing condition along $r^{(1)} = 0,1$ and $r^{(2)} = 0,1$ as
\begin{equation}\label{bdry1_curvi}
\mathcal{T}\cdot{\bf n}_1^{\pm} = \frac{\pm 1}{J\left|\nabla r^{(1)}\right|}\bar{A}_1\bar{\nabla}{\bf u}, \ \ \ r^{(1)} = 1, 0,
\end{equation}
\begin{equation}\label{bdry2_curvi}
\mathcal{T}\cdot{\bf n}_2^{\pm} = \frac{\pm 1}{J\left|\nabla r^{(2)}\right|}\bar{A}_2\bar{\nabla}{\bf u}, \ \ \ r^{(2)} = 1, 0,
\end{equation} 
respectively. Here, the ${\bf n}_1^{\pm}$ and ${\bf n}_2^{\pm}$ have a similar definition as ${\bf n}_3^{\pm}$ in (\ref{definition_bdry3_curvi}).
