%!TEX root = elastic_3d_sbp.tex
\subsection{The Cartesian coordinates}
The problem is defined on the domain ${\bf x}\in\Omega = [0,a^{(1)}]\times[0,a^{(2)}]\times[0,a^{(3)}]$ with ${\bf x} = (x^{(1)},x^{(2)},x^{(3)})^T$ are Cartesian coordinates. Denote ${\bf u} = (u_1,u_2,u_3)^T$ to be the three dimensional displacement vector in Cartesian coordinates, then the elastic wave equation takes the form,
\begin{align*}
\rho\frac{\partial^2{\bf u}}{\partial^2 t} &= \nabla\cdot\mathcal{T} + {\bf F}, \ \ \ {\bf x}\in\Omega,\ \ \ t\geq 0,\\
\nabla\cdot\mathcal{T} &:= L{\bf u},
\end{align*}
provided with appropriate initial and boundary conditions, especially, we consider periodic boundary conditions in the directions $1$ and $2$ for the rest of the paper and the boundary conditions in the direction $3$ will be given later. Here, $\rho$ is density, $\mathcal{T}$ is the stress tensor and ${\bf F}$ is the force function. The spatial operator $L$ is called $3\times3$ symmetric Kelvin-Christoffel differential operator matrix, specifically,
\begin{equation*}
L{\bf  u} = \partial_1(A_1\nabla{\bf u}) + \partial_2(A_2\nabla{\bf u}) + \partial_3(A_3\nabla{\bf u}),
\end{equation*}
with
\begin{align*}
A_1\nabla{\bf u} &:= M_{11}\partial_1{\bf u} + M_{12}\partial_2{\bf u} + M_{13}\partial_3{\bf u}, \\
A_2\nabla{\bf u} &:= M_{21}\partial_1{\bf u} + M_{22}\partial_2{\bf u} + M_{23}\partial_3{\bf u}, \\
A_3\nabla{\bf u} &:= M_{31}\partial_1{\bf u} + M_{32}\partial_2{\bf u} + M_{33}\partial_3{\bf u},
\end{align*}
where $M_{ij}, i,j = 1,2,3$ are defined by
\begin{equation}\label{Mmatrices}
M_{ij} = P^T_iCP_j.
\end{equation}
Here, $C$ is symmetric and positive definite, we refer to Appendix ? for the definitions of matrices $C$ and $P_i, i = 1,2,3$. For the matrices $M_{ij}$, we have that $M_{ii}$ are symmetric positive definite, and $M_{ij}=M^T_{ji}$, for $i,j=1,2,3$. Especially, for the isotropic elastic wave equation, we have
\[ M_{11} = \left(\begin{array}{ccc}
2\mu+\lambda & 0 & 0\\
0 & \mu & 0\\
0 & 0 & \mu\end{array}\right), M_{12} = \left(\begin{array}{ccc}
0 & \lambda & 0\\
\mu & 0 & 0\\
0 & 0 & 0\end{array}\right), M_{13} = \left(\begin{array}{ccc}
0 & 0 & \lambda\\
0 & 0 & 0\\
\mu & 0 & 0\end{array}\right),\]
\[ M_{21} =(M_{12})^T, M_{22} = \left(\begin{array}{ccc}
\mu & 0 & 0\\
0 & 2\mu+\lambda & 0\\
0 & 0 & \mu\end{array}\right), M_{23} = \left(\begin{array}{ccc}
0 & 0 & 0\\
0 & 0 & \lambda\\
0 & \mu & 0\end{array}\right),\]
\[ M_{31} = (M_{13})^T, \ \ \ \ \ M_{32} =(M_{23})^T, \ \ \ \ \ M_{33} = \left(\begin{array}{ccc}
\mu & 0 & \lambda\\
0 & \mu & 0\\
0 & 0 & 2\mu+\lambda\end{array}\right),\]
Here, $\lambda$ and $\mu$ are the first and second Lame parameters respectively, which are determined by the properties of the materials.

Denote the unit outward normal ${\bf n}_i^{\pm} = (n_i^{\pm,(1)},n_i^{\pm,(2)},n_i^{\pm,(3)})$ for the boundaries $x^{(i)} = 0, a^{(i)}, i = 1,2,3$ respectively. For example, ${\bf n}_1^{\pm} = (\pm 1, 0,0)$ for the boundaries $x^{(1)} = a^{(1)}$ and $x^{(1)} = 0$ respectively, then we have the boundary traction forcing 
\begin{equation*}
{\bf n}_i^{\pm}\cdot\mathcal{T} = n^{\pm,(1)}_iA_1\nabla{\bf u} + n^{\pm,(2)}_iA_2\nabla{\bf u} + n^{\pm,(3)}_iA_3\nabla{\bf u},\ \ i = 1,2,3.
\end{equation*}
A homogeneous Dirichlet boundary condition corresponds to ${\bf u} = {\bf 0}$ and a free surface boundary condtion is ${\bf n}_i^{\pm}\cdot\mathcal{T}  = {\bf 0}, i = 1,2,3$.

It is known that the elastic wave equation with the homogeneous Dirichelet boundary conidtion or free surface boundary condition in a Cartesian domain is a well-posed problem and the total energy of the solution is conserved if the external force ${\bf F} = 0$, we refer to \cite{?} for a detailed analysis.
