\documentclass[a4paper]{article}

\usepackage[english]{babel}
\usepackage[utf8]{inputenc}
\usepackage{amsmath}
\usepackage{graphicx}
\usepackage[colorinlistoftodos]{todonotes}
\usepackage{tikz}
\usetikzlibrary{arrows}
\usepackage{booktabs}
\usepackage{threeparttable}
\usepackage{tikz}
\usetikzlibrary{arrows.meta}
\usepackage{pgfplots}
\usepackage{subcaption}
\usepackage[toc,page]{appendix}

\title{Fourth order summation-by-parts finite difference method for wave propagation in anisotropic elastic material and curvilinear coordinates with mesh refinement interface}

\date{\today}

\begin{document}
\maketitle

\begin{abstract}
We analyze
\end{abstract}

\section{Introduction}

\section{The anisotropic elastic wave equation}
We consider the anistropic elastic wave equation in three dimensional domain ${\bf x} \in \Omega$, ${\bf x} = (x_1,x_2,x_3)^T$ are Cartesian coordinates. Denote ${\bf u} = (u_1,u_2,u_3)^T$ to be the three dimensional displacement vector in Cartesian coordinates, then the elastic wave equation in Cartisian coordinates takes the form,
\begin{eqnarray*}
    \rho\frac{\partial^2{\bf u}}{\partial^2 t} &=& \nabla\cdot\mathcal{T} + {\bf F}, \ \ \ {\bf x}\in\Omega,\ \ \ t\geq 0,\\
    \nabla\cdot\mathcal{T} &:=& {\bf Lu},
\end{eqnarray*}
provided with appropriate initial and boundary conditions. Here, $\rho$ is density, $\mathcal{T}$ is stress tensor and ${\bf F}$ is the force function. The spatial operator ${\bf L}$ is called $3\times3$ symmetric Kelvin-Christoffel differential operator matrix, specifically,
\begin{equation*}
    {\bf L u} = \partial_1(A_1\nabla{\bf u}) + \partial_2(A_2\nabla{\bf u}) + \partial_3(A_3\nabla{\bf u}),
\end{equation*}
with
\begin{eqnarray*}
A_1\nabla{\bf u} &:=& M^{11}\partial_1{\bf u} + M^{12}\partial_2{\bf u} + M^{13}\partial_3{\bf u}, \\
A_2\nabla{\bf u} &:=& M^{21}\partial_1{\bf u} + M^{22}\partial_2{\bf u} + M^{23}\partial_3{\bf u}, \\
A_3\nabla{\bf u} &:=& M^{31}\partial_1{\bf u} + M^{32}\partial_2{\bf u} + M^{33}\partial_3{\bf u},
\end{eqnarray*}
where $M^{i,j}, i = 1,2,3, j = 1,2,3$ are determined by the material properties. For example, for isotropic elastic material,
\[ M^{11} = \left(\begin{array}{ccc}
2\mu+\lambda & 0 & 0\\
0 & \mu & 0\\
0 & 0 & \mu\end{array}\right), M^{12} = \left(\begin{array}{ccc}
0 & \lambda & 0\\
\mu & 0 & 0\\
0 & 0 & 0\end{array}\right), M^{13} = \left(\begin{array}{ccc}
0 & 0 & \lambda\\
0 & 0 & 0\\
\mu & 0 & 0\end{array}\right),\]




\subsection{Energy estimate}

\section{Generalization to curvilinear coordinates}

\subsection{Boundary Condition}

\subsection{Energy Estimate}

\section{Grid refinement interface}

\section{Numerical Experiments}

\end{document}
